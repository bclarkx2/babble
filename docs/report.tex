% Class

\documentclass{article}


% Info

\title{Babble: Learning Grammar Structure with Generative Adversarial Networks}
\author{Brian Clark, CWRU}

\date{\today}

% Packages

% \usepackage{amsmath}
% \usepackage{nccmath}
% \usepackage{enumerate}
% \usepackage{bm}
% \usepackage{textcomp}
% \usepackage{numproof}
% \usepackage{math_shortcuts}
\usepackage{blindtext}
\usepackage{graphicx}
\usepackage{cite}
\usepackage{hyperref}

% Settings

\setlength{\parindent}{0pt}
\setlength{\parskip}{1em}
\graphicspath{ {img/} }


% Commands


% Document

\begin{document}
\maketitle
\tableofcontents

\abstract{This is the text of the abstract}

\section{Introduction}

\subsection{Problem}
Goal: Generate text that looks like it came from a language
Problems: How to determine whether the text is from the language?

\subsection{Grammar}
    Rule system to describe language
    Defined grammar -> any sentence can be judged
    If we had grammar, could use as penalty function
    Complexity
    Rather than come up with rules, rely on data

\subsection{Generative Adversarial Networks}
    How they work
    Role of generator
    Role of discriminator

\section{Background}
    ???

\section{Approach}
    Description
        role of generator
            inputs
            outputs
        role of discriminator
            inputs
            outputs
        overall diagram
    Grammar definition
        Abstract definition
            class diagram?
        Tools
            nltk
        Grammars
            SimpleGrammar
                description
                generation process
            \_Grammar
            English
    Network definition
        Tools
            keras
        Generator structure
            diagram
        Discriminator structure
            diagram

\section{Results}
    Experiments
        Network structure
        Amount of data

\section{Discussion}
\blindtext

\section{Conclusions}


Some text, from \cite{chollet2015keras}

\bibliography{sources}{}
\bibliographystyle{plain}

\end{document}
